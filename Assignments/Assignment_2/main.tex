%%%%%%%%%%%%%%%%%%%%%%%%%%%%%%%%%%%%%%%%%%%%%%%%%%%%%%%%%%%%%%%%%%%%%%%%%%%%%%%%
% Christian's Assignment Latex Template
%
% Inspired by Simon Pratt's LatexTemplate (github.com/spratt/LatexTemplate)
% Modified by Christian Delahousse (christian.delahousse.ca)
%%%%%%%%%%%%%%%%%%%%%%%%%%%%%%%%%%%%%%%%%%%%%%%%%%%%%%%%%%%%%%%%%%%%%%%%%%%%%%%%

\documentclass[12pt]{article}


\usepackage{styles/Assignment}
\usepackage{amssymb}
\usepackage{graphicx}
\usepackage{amsmath}
\usepackage{fancybox}

\newcommand{\IN}{\mathbb{N}}
\newcommand{\IZ}{\mathbb{Z}}

%%%%%%%%%%%%%%%%%%%%%%%%%%%%%%%%%%%%%%%%%%%%%%%%%%%%%%%%%%%%%%%%%%%%%%%%%%%%%%%%
% Assignment
%%%%%%%%%%%%%%%%%%%%%%%%%%%%%%%%%%%%%%%%%%%%%%%%%%%%%%%%%%%%%%%%%%%%%%%%%%%%%%%%

\CourseAndAssignmentName{COMP2804}{Assignment 2}
\StudentNameAndNumber{Seena Rowhani}{100945353}

\begin{document}
%====================QUESTION 1 - Done ================

\begin{question}
See the top of the page.
\end{question}

%====================QUESTION 2 - Done ================
\begin{question} \\ \\
The functions $f : \IN \rightarrow \IN$ and $g : \IN \rightarrow \IN$ 
are recursively defined as follows:
\[ \begin{array}{lcll} 
      f(0) & = & 1 , & \\ 
      f(n) & = & g(n,f(n-1)) & \mbox{if $n \geq 1$,}  \\ 
      g(m,0) & = & 0 & \mbox{if $m \geq 0$,} \\ 
      g(m,n) & = & m + g(m,n-1) & \mbox{if $m \geq 0$ and $n \geq 1$.}
   \end{array} 
\] 
Solve these recurrences for $f$, i.e., express $f(n)$ in terms of $n$.  
Justify your answer. 
\end{question}
%==================Answer================
\begin{answer}
It's evident that the solution, expressed in terms of $n$, is $n!$. \textbf{This can be said because the function \textit{g(m,n)} will add \textit{m} to itself \textit{n} times, for every value of $n$ that occurs}.\newline I will prove that this is correct through induction. Since I say I can express $f(n)$ as $n!$, I will show that $f(n+1) = (n+1)!$ \\ 

\textbf{But!} Before we do this, I must show that what I had stated about $g_{m,n} = m*n$ is indeed true. So before I make any statements about $f_{n}$. \\

\textbf{Base Case} 
\begin{align*}
m&=1, n=2 \\
g_{1,1} &= 1 + g_{1,0} \\
        &= 1 + 0 = 1
\end{align*}

\textbf{Inductive Hypothesis} \\
Assume that,
$$ g_{m,n} = m\cdot n $$
Let an integer, $k$ be equal to $n$\\ \\
\textbf{Inductive Step}\\
If I can show that $g_{m,k+1} = m\cdot(n+1)$, then I have proved this to be true. 
\begin{align*}
g_{m,k+1} &= m + g_{m, k} \\
          &= m + m \cdot k \\
          &= m \cdot (k+1)
\end{align*}
Because I have shown that this is true, I can proceed to my proof on function $f$. \\ \\
\textbf{Base Case}
$$f(0) = 1$$
\textbf{Inductive Hypothesis} \\ For some integer $k$\dots $$f(k) = k!$$ If it can be argued that this is true for n, it must hold that it will be true for all values of $n$ that follow\dots \\ \\ \textbf{Inductive Step} \\ If it can be shown that\dots $$f(k+1) = (k+1)!$$ \dots then my proof is complete.
\begin{align*}
f(k+1) 
&= g(k+1,f(k))\\
&= (g(k+1, k!))\\
&= (k+1)k! \\
&= (k+1)! \\
\end{align*}
Therefore, through mathematical induction has it become evident that the recursive definition of $f_{n}$ is simply $n!$.
\end{answer}


%====================QUESTION 3 - Done==================
\begin{question} \\ \\
The function $f : \IN \rightarrow \IZ$ is defined by 
\[ f(n) = 2n ( n-6) 
\]
for each integer $n \geq 0$. Derive a recursive form of this 
function $f$. Show your work. 
\end{question} 

\begin{answer}
	$f_{n} = 2n(n-6)$ can be expressed  as $f_{n} = (f_{n} + f{n-1}) - f_{n-1}$ allows us to express it in it's recursive form, so to speak \dots \\
	
	\textbf{Base Case} 
	\\
	$$ f_{1} = 1 $$
	
	
	\begin{align*}
	 f_{n} &= 2n(n-6) \\
	 f_{n - 1} &= 2(n-1)( (n - 1) - 6 )\\
	 f_{n} &= f_{n} + f_{n-1} - f_{n-1} \\
	 f_{n} &= \{ 2n(n-6) + 2(n-1)( (n - 1) - 6 ) \} - f_{n-1} \\
	\end{align*}
	
This statement algebraically proves itself to be true, as I have essentially just added $0$ to the original function (which does not affect its value).
\end{answer}



%====================QUESTION 4 - Done===================
\begin{question}  \\ \\
The function $f : \IN^3 \rightarrow \IN$ is defined as follows: 
\begin{align*}
      f(k,n,0) &= k+n  \\
      f(k,0,1) &= 0\\
      f(k,0,2) &= 1  \\
      f(k,0,i) &= k \\
      f(k,n,i) &= f ( k , f(k,n-1,i) , i-1)
\end{align*}
Determine $f(2,3,2)$. Show your work. 
\end{question} 

\begin{answer}
\begin{align*}
f(2,3,2) &= f(2, f(2, 2, 2), 1) &_{f(2,2,2)}\\
         &= f(2, 4, 1)  &_{f(2,4,1)}\\
         &= 8
\end{align*}

\begin{align*}
f(2,2,2) &= f(2, f(2,1,2), 1)  &_{f(2,1,2)}\\
         &= f(2,2,1)\\
         &= f(2, f(2,1,1),0) \\
         &= 2 + f(2,1,1) \\
         &= 2 + f(2, f(2,0,1), 0) \\
         &= 2 + f(2,0,0) \\
         &= 2+ 2 \\
         &= 4 \\
\end{align*}
\begin{align*}
f(2,1,2) &= f(2, f(2,0,2),1)  &\\
         &= f(2,1,1) \\
         &= f(2, f(2,0,1), 0) \\
         &= 2
\end{align*}
\begin{align*}
f(2,4,1) &= f(2, f(2,3,1), 0) \\
         &= 2 + f(2,3,1) \\
         &= 2 + f(2, f(2,2,1), 0) \\
         &= 2 + 2 + f(2, f(2,1,1), 0) \\
         &= 2 + 2 + 2 + f(2, f(2,0,1), 0) \\
         &= 2 + 2 + 2 + f(2,0,0) \\
         &= 2 + 2 + 2 + 2 \\
         &= 8
\end{align*}
Therefore, the function call of $f(2,3,2)$ will return $8$\dots
\end{answer}

%===============QUESTION 5 - Done================
\begin{question} \\ \\ 
A \emph{maximal run of ones} in a bitstring is a maximal consecutive
substring of ones. For example, the bitstring $1100011110100111$
has four maximal runs of ones: $11$, $1111$, $1$, and $111$. These 
maximal runs have lengths $2$, $4$, $1$, and $3$, respectively. 

Let $n \geq 1$ be an integer and let $B_n$ be the number of bitstrings 
of length $n$ that do not contain any maximal run of ones of odd 
length; in other words, every maximal run of ones in these bitstrings 
has an even length.\\ \\
\parbox{\linewidth}{
\begin{itemize} 
\item Determine $B_1$, $B_2$, and $B_3$. 
\item Determine the value of $B_n$, i.e., express $B_n$ in terms of 
      numbers that we have seen in class. Justify your answer. 
      \emph{Hint:} Derive a recurrence. 
\end{itemize}}
\end{question} 

\begin{answer}
\begin{align*}
B_{1} &= 1 \\
B_{2} &= 2 \\
B_{3} &= 3 \\
B_{4} &= 5 \\
B_{n} &= B_{n-1} + B_{n-2}
\end{align*}

If you put all of the strings of length $n$ in a matrix that does not contain any odd runs of $1$'s. The number of rows is going to be $B_{n}$, the number of columns will be $n$, and you would divide the matrix into two. If we fix the first bit in the first half to be 1, then you'd have bit strings of length $n-2$ left. \newline 

For the bottom half, if we fix the first bit to be $0$, there would be $n-1$ characters to choose, therefore $B_{n} = B_{n-1} + B_{n-2}$. This is just \textbf{fibonacci}, but misplaced by one index. \\

Therefore I can define the following: 
$$B_{n} = fibonacci_{(n+1)}$$
\end{answer}

%===============QUESTION 6 - Done ================

\begin{question} \\ \\
Let $n \geq 1$ be an integer and let $S_n$ be the number of ways in which 
$n$ can be written as a sum of $1$s and $2$s; the order in which the 
$1$s and $2$s occur in the sum matters. For example, $S_3 = 3$, because 
\[ 3 = 1 + 1 + 1 = 1 + 2 = 2 + 1 . 
\] \\
\parbox{\linewidth}{
\begin{itemize} 
\item Determine $S_1$, $S_2$, and $S_4$. 
\item Determine the value of $S_n$, i.e., express $S_n$ in terms of 
      numbers that we have seen in class. Justify your answer. 
      \emph{Hint:} Derive a recurrence. 
\end{itemize}}
\end{question} 
%==================Answer================
\begin{answer}
\begin{align*}
S_{1} &= 1 \\
S_{2} &= 2 \\
S_{3} &= 3 \\ 
S_{4} &= 5 \\
S_{5} &= 8 \\
S_{n} &= S_{n-1} + S_{n-2}
\end{align*}

Through the sample, it is also evident that the number of ways to rearrange the 1's and 2's is equivalent to the Fibonacci recursion. \\

This is because every large block can be reduced into a subset of options. For example, $S_{3} = S_{2} + S{1} = 2 + 1 = 3$. This pattern can be repeated for any $n$, as each is composed 
\end{answer}

%===============QUESTION 7 - Done ================

\begin{question} \\ \\
Let $n$ be a positive integer and consider a $5 \times n$ board $B_n$
consisting of $5n$ cells, each one having sides of length one. The top
part of the figure below shows $B_{12}$.


A \emph{brick} is a horizontal or vertical board consisting of 
$2 \times 3 = 6$ cells; see the bottom part of the figure above.
A \emph{tiling} of the board $B_n$ is a placement of bricks on the
board such that \\ \\
\parbox{\linewidth}{
\begin{itemize}
\item the bricks exactly cover $B_n$ and
\item no two bricks overlap.
\end{itemize}}
\\
The figure below shows a tiling of $B_{12}$.

Let $T_n$ be the number of different tilings of the board $B_n$. \\  \\
\parbox{\linewidth}{
\begin{itemize} 
\item Let $n \geq 6$ be a multiple of $6$. Determine the value of $T_n$.
      Justify your answer. \emph{Hint:} Derive a recurrence.  
\item Let $n$ be a positive integer that is not a multiple of $6$. 
      Prove that $T_n = 0$. 
\end{itemize}
}
\end{question} 

\begin{answer}
$n$ must be $\geq 6$, where $ n \% 6 = 0$. So in other words, $n$ must be completely divisible by 6. \\
Our \textbf{base case} becomes \dots
$$ f(6) = 2 $$

For each block of 6, there are only \textbf{two ways they can be arranged}. This is because of the combinations of blocks that are 2 wide, and of those that are three wide, there is no combination of 2 + 3 that will add up to 6. Therefore, there are only two combinations of each. The combination of \textbf{two} three length blocks, and \textbf{three} two length blocks.\\

Using the product rule we would have 2 * 2 * ... for every block that we have. We can express this in terms of n by describing each block as $\frac{1}{6}$ to the number of columns $n$\dots 
\\
Therefore there are:
$$ 2^{n/6} $$ number of ways to arrange the tiles, \textit{for an $n$ which matches the above conditions}. \\

We can express this as a recursion as follows:

$$ T_{n} = 2\cdot T_{n-6} \qquad if(n\geq 6 \land n\%6==0) = 2^{n/6}$$

This can be proven through mathematical \textbf{induction}\dots \\

\textbf{Base Case}
$$T_{6} = 6$$

\textbf{Inductive Hypothesis} \\
Let an integer, $k$, be equal to our $n$. If it can be shown that our function holds for a value $k+6$ \textit{(since every number has to be a multiple of 6)}, then it must hold it would work for all values of $k$ that are divisible by 6, and will therefore be true.  
\begin{align*}
T_{k}
&= 2 * T_{n-6} \\
&= 2^{k/6}
\end{align*}

\textbf{Inductive Step}
\begin{align*}
T_{k+6} &= 2 T_{k} \\
        &= 2\cdot 2^{(k/6)} \\
        &= 2^{k/6 + 1}
\end{align*}

Therefore, it can be seen that through the principles \textbf{mathematical induction} that for every integer, $k$ that is a multiple of $6$, the recursion, 
$$T_{k} = 2\cdot T_{(k-6)}$$ holds indefinitely throughout the series.\\
\end{answer}
%====================QUESTION 8 - Done==================
%Silly
% 2*log2(n) - 1 for n thats a power of 2
% silly(n) = silly(n/2) + n/2

\begin{question}\\ \\
Consider the following recursive algorithm $silly$, which takes as 
input an integer $n \geq 1$ which is a power of $2$:

\begin{quote}
\begin{tabbing}
{\bf if} $n=1$ \\
{\bf then} drink one pint of beer \\
{\bf else} \= {\bf if} $n=2$ \\
           \> {\bf then} fart once \\
           \> {\bf else} \= fart once; \\
           \>            \> $silly(n/2)$; \\
           \>            \> fart once \\
           \> {\bf endif} \\
{\bf endif}
\end{tabbing}
\end{quote}

For $n$ a power of $2$, let $F(n)$ be the number of times you fart when
running algorithm $silly(n)$. 
\end{question}
\begin{itemize}
\item Determine the value of $F(n)$ and prove that your answer is correct. 
      \emph{Hint:} Derive a recurrence. 
\end{itemize} 

%==================Answer 8 ======================
\begin{answer}
For every value of $n$, in which $n$ is equal to $2^{k}$ for some integer $k \geq 1$. To start, I must first witness a pattern\dots
\begin{align*}
Silly_{1}  &= 0\\
Silly_{2}  &= 1\\
Silly_{4}  &= 3\\
Silly_{8}  &= 5\\
Silly_{16} &= 7\\
Silly_{32} &= 9
\end{align*}

I can show that the following recurrence will hold\dots

\begin{align*}
Silly_{(n)} &= Silly_{n/2} + 2 \\
Silly_{(8)} &= Silly_{(4)} + 2 \\
            &= Silly_{(2)} + 2 + 2 \\
            &= Silly_{(1)} + 1 + 2 + 2 \\
            &= 0 + 1 + 2 + 2 \\
            &= 5
\end{align*}

I will now argue that $Silly_{n}$ can be expressed in terms of $n$ as:

$$ Silly_{(n)} = 2log_{2}(n) - 1$$

I can prove this to be true via induction \dots \\

\textbf{Base Case}
\begin{align*}
    Silly_{1} = 0 \\
    Silly_{2} = 1 \\
\end{align*}

\textbf{Inductive Hypothesis}\\
Let $k$ be an integer, where\\ $k = n$ and $log_{2}(k) \in \emph{Z}$. \\
Assume, 
$$ Silly_{(k/2)} = 2\cdot log_{2} ((k/2)) - 1 $$
\dots to be true.\\
\textbf{Inductive Step}
If it can be shown that, 
$$Silly_{(k)} = Silly_{(k/2)} + 2 $$ then it should hold for all powers of two
\begin{align*}
Silly_{(k)}  &= Silly_{(k/2)} + 2\\
             &= (2\cdot log_{2}(k/2) -1 )+ 2 \\
             &= 2\cdot (log_{2}(k) - log_{2}(2) ) + 1 \\
             &= 2\cdot (log_{2}(k) - 1 ) + 1 \\
             &= 2\cdot log_{2}(k) + 1 - 2 \\
             &= 2\cdot log_{2}(k) - 1 \\
             &= Silly_{(k)}
\end{align*}
Therefore, it can be seen, through mathematical induction, that for all values of $n$, greater or equal to 1, as well as a power of 2 will follow the aforementioned formula in terms of $n$.
\end{answer}

%=================Question 9 ========================

\begin{question} \\ \\
Let $m \geq 1$ and $n \geq 1$ be integers. Consider $m$ horizontal lines 
and $n$ non-horizontal lines such that  
\begin{itemize} 
\item no two of the non-horizontal lines are parallel, 
\item no three of the $m+n$ lines intersect in one single point. 
\end{itemize} 
These lines divide the plane into regions (some of which are bounded and 
some of which are unbounded). Denote the number of these regions by 
$R_{m,n}$. From the figure below, you can see that $R_{4,3} = 23$. 


\begin{itemize} 
\item Derive a recurrence for the numbers $R_{m,n}$ and use it to prove 
      that 
      \[ R_{m,n} = 1 + m(n+1) + {{n+1} \choose 2} .  
      \] 
\end{itemize} 
\end{question}

%===============Answer=========================

\begin{answer}
When you remove a line, $n$, you decrease the number of regions by $(m+n)$. This is because for every n, it will intersect with all non horizontal and all horizontal lines; \textbf{because no two lines can be parallel}.\\
Therefore the recursion can be encapsulated by the following expression:
$$R_{m,n} = R_{m,n-1} + m + n $$
To quickly verify that this recursion is accurate, we can test it against the given derivation in terms of $n$
\begin{align*}
R_{2, 2} &= R_{2,1} + 2 + 2 \\
         &= R_{2,0} + 2 + 1 + 4 \\
         &= 2 + 1 + 7 \\
         &= 10 \\
R_{2,2} &= 1 + 2(2+1) + {{2+1} \choose 2} \\
        &= 1 + 6 + 3 \\
        &= 10
\end{align*}

We can prove that this is true using induction, to show that if it indeed works for an assumed hypothesis, it should logically follow that all elements after that for some $n+1$. If we can show that this is true, then the statement is \textbf{completely rendered true} \\

\textbf{Base Case}
   $$ R_{m, 0} &= m + 1$$
   
\textbf{Inductive Hypothesis} \\
Let an integer $k = n$, and assume that the following equality is true for all values of $k,m \geq 1$:
$$R_{m,k} = R_{m,k-1} + m + k$$ 
If I can express $R_{m,k} = 1 + m(k+1) { {k+1} \choose 2}$, then my hypothesis will be proven correct \dots \\
\textbf{Inductive Step} 
\begin{align*}
    R_{m,k} &= R_{m,k-1} + m + k \\
            &= (1 + m\cdot k + {k \choose 2}) + m + k \\
            &= (1 + m\cdot k + \frac{k(k-1)}{2}) + m + k\\
            &= 1 + m(k+1) + \frac{k(k-1)}{2} + \frac{2k}{2}\\
            &= 1 + m(k+1) + \frac{k^{2} - k + 2k}{2} \\
            &= 1 + m(k+1) + \frac{k^{2} + k}{2} \\
            &= 1 + m(k+1) + \frac{k(k+1)}{2} \\
            &= 1 + m(k+1) + { {k+1} \choose 2 } \\
            &= R_{m,k}
\end{align*}
Therefore, it is evident through mathematical induction, that $R_{m,k}$ can indeed be expressed in the aforementioned recursion.
\end{answer}


\end{document}